%! LuaLaTeX 文書
\documentclass[jafontsize=12pt]{jlreq}

\usepackage{luatexja}
\ltjdefcharrange{11}{`→,`↑,`↓,`←}
\ltjsetparameter{jacharrange={-2,-8,+11}}
\usepackage[no-math,match,deluxe,jfm_yoko=jlreq]{luatexja-preset}
\usepackage{luatexja-otf,luatexja-adjust}
\newopentypefeature{PKana}{On}{pkna}

\usepackage{yhmath,amssymb,mathtools,mathabx,mathrsfs,mathbbol}

\usepackage[math]{iwona}
\usepackage[euler-digits]{eulervm}
\usepackage{yhmath}
\usepackage[scaled]{beramono}
\DeclareMathAlphabet{\mathtt}{T1}{fvm}{m}{n}
\DeclareMathAlphabet{\mathsf}{T1}{uop}{m}{n}
%% フォントがない場合は以下の5行を削除
\setsansjfont[Ligatures=TeX,BoldFont=RodinNTLGPro-B]{FOT-RodinNTLGPro-DB}
\setmainjfont[Ligatures=TeX,BoldFont=RodinNTLGPro-B]{FOT-RodinNTLGPro-DB}
\setsansfont[Ligatures=TeX,BoldFont=RodinNTLGPro-B]{FOT-RodinNTLGPro-DB}
\setmainfont[Ligatures=TeX,BoldFont=RodinNTLGPro-B]{FOT-RodinNTLGPro-DB}
\ltjsetparameter{yjabaselineshift=0pt,yalbaselineshift=0.5pt}

\usepackage{scalefnt}
\usepackage{multirow}
\usepackage{multicol}
\ltjenableadjust[lineend=extended,priority=true,profile=true,linestep=true]
\allowdisplaybreaks[4]

\usepackage{hmtrump}

\usepackage[a4paper,margin=20mm]{geometry}

\newcommand{\cellalign}[2]{\multicolumn{1}{#1}{#2}}

%%%%%%%%%%%%%%%%%%%%%%%


\begin{document}

\pagestyle{empty}

\begin{center}
{\LARGE オルナファザールマニ ルール早見}\\ひとみさん \today
\end{center}

\setlength{\parindent}{0pt}

\section{ゲームの流れ}
\textbf{\mbox{シャッフル}\hfill
→\hfill\mbox{カット(右隣)}\hfill
→\hfill\mbox{配る}\hfill
→\hfill\mbox{交換}\hfill
→\hfill\mbox{トリックテイキング}\hfill}\\
カードをカットした結果、底になったカードでゲームの種類を決定\\
配り方: マニ(中央) 4 → 左隣から順に 3 を 4 回

\section{ゲームの種類/得点}
\begin{table}[h]
\centering
\caption{ゲームの種類}
\begin{tabular}{cl}
\hline
\cellalign{c}{底になったカード}&\cellalign{c}{ゲームの目的}\\
\hline\hline
\trumpx A \trumpx K \trumpx Q \trumpx J \trumpx T&切札なしでたくさんのトリックを取る\\
\trumpx 9 \trumpx 8 \trumpx 7 \trumpx 6&そのスートを切札にしてたくさんのトリックを取る\\
\trumpx 5 \trumpx 4 \trumpx 3 \trumpx 2&トリックをなるべく取らない (Nolo)\\
\hline
\end{tabular}
\end{table}
\rule{0pt}{0pt}\hfill \textbf{{\Large 目標: 4トリック}}\hfill\rule{0pt}{0pt}

トリックをたくさん取る→4トリックより多いぶんプラス得点、少ないぶんマイナス得点\\
ノロ→4トリックより多いぶんマイナス得点、少ないぶんプラス得点

\section{カードの交換}
\begin{table}[h]
\centering
\begin{tabular}{cc}
ディーラーの左{\small (オープニングリード)}&7 枚まで\\
ディーラーの右&5 枚まで\\
ディーラー&残ってる枚数まで
\end{tabular}
\end{table}


\end{document}